% Définitions pour le glossaire
% À inclure avec % Définitions pour le glossaire
% À inclure avec % Définitions pour le glossaire
% À inclure avec % Définitions pour le glossaire
% À inclure avec \input{glossaire} dans le préambule du document principal

\newglossaryentry{lwr}{
  name={Modèle LWR},
  description={Modèle macroscopique du trafic développé indépendamment par Lighthill et Whitham et par Richards. Il décrit l'évolution du trafic routier en utilisant une équation de conservation et une relation fondamentale entre densité, vitesse et flux}
}

\newglossaryentry{zemidjans}{
  name={Zémidjans},
  description={Moto-taxis au Bénin, constituant un mode de transport essentiel et une part importante du trafic routier urbain et péri-urbain}
}

\newglossaryentry{multiclasse}{
  name={Modèle multiclasses},
  description={Extension des modèles de trafic distinguant différentes catégories de véhicules, chacune avec ses propres caractéristiques (vitesse libre, taille, comportement)}
}

\newglossaryentry{gapfilling}{
  name={Gap-filling},
  description={Comportement caractéristique des motos consistant à occuper les espaces entre les véhicules plus grands, augmentant ainsi la densité effective du trafic}
}

\newglossaryentry{interweaving}{
  name={Interweaving},
  description={Comportement de circulation en zigzag des motos entre les files de véhicules, créant un flux transversal qui peut perturber l'écoulement des autres classes}
}

\newglossaryentry{coefralentissement}{
  name={Coefficient de ralentissement},
  description={Paramètre $\lambda_{\text{mat},i}$ qui module la vitesse maximale d'une classe de véhicule en fonction du type de revêtement routier}
}

\newglossaryentry{diagrammefondamental}{
  name={Diagramme fondamental},
  description={Représentation graphique de la relation entre la densité, la vitesse et le flux du trafic routier}
}

\newglossaryentry{ondechoc}{
  name={Onde de choc},
  description={Discontinuité se propageant dans le flux du trafic, généralement associée à la transition entre différents régimes de circulation}
}

\newglossaryentry{densitecritique}{
  name={Densité critique},
  description={Valeur de densité correspondant au flux maximal, marquant la transition entre régime fluide et congestionné}
}

\newglossaryentry{calibration}{
  name={Calibration},
  description={Processus d'ajustement des paramètres d'un modèle pour reproduire au mieux les données observées}
}

\newglossaryentry{validation}{
  name={Validation},
  description={Processus d'évaluation de la qualité et de la pertinence d'un modèle à l'aide de données indépendantes de celles utilisées pour la calibration}
}

\newglossaryentry{rmse}{
  name={RMSE},
  description={Root Mean Square Error (Erreur Quadratique Moyenne), métrique utilisée pour évaluer la précision d'un modèle}
}

\newglossaryentry{geh}{
  name={GEH},
  description={Statistique de Geoffrey E. Havers couramment utilisée pour comparer les flux de trafic observés et modélisés}
}
 dans le préambule du document principal

\newglossaryentry{lwr}{
  name={Modèle LWR},
  description={Modèle macroscopique du trafic développé indépendamment par Lighthill et Whitham et par Richards. Il décrit l'évolution du trafic routier en utilisant une équation de conservation et une relation fondamentale entre densité, vitesse et flux}
}

\newglossaryentry{zemidjans}{
  name={Zémidjans},
  description={Moto-taxis au Bénin, constituant un mode de transport essentiel et une part importante du trafic routier urbain et péri-urbain}
}

\newglossaryentry{multiclasse}{
  name={Modèle multiclasses},
  description={Extension des modèles de trafic distinguant différentes catégories de véhicules, chacune avec ses propres caractéristiques (vitesse libre, taille, comportement)}
}

\newglossaryentry{gapfilling}{
  name={Gap-filling},
  description={Comportement caractéristique des motos consistant à occuper les espaces entre les véhicules plus grands, augmentant ainsi la densité effective du trafic}
}

\newglossaryentry{interweaving}{
  name={Interweaving},
  description={Comportement de circulation en zigzag des motos entre les files de véhicules, créant un flux transversal qui peut perturber l'écoulement des autres classes}
}

\newglossaryentry{coefralentissement}{
  name={Coefficient de ralentissement},
  description={Paramètre $\lambda_{\text{mat},i}$ qui module la vitesse maximale d'une classe de véhicule en fonction du type de revêtement routier}
}

\newglossaryentry{diagrammefondamental}{
  name={Diagramme fondamental},
  description={Représentation graphique de la relation entre la densité, la vitesse et le flux du trafic routier}
}

\newglossaryentry{ondechoc}{
  name={Onde de choc},
  description={Discontinuité se propageant dans le flux du trafic, généralement associée à la transition entre différents régimes de circulation}
}

\newglossaryentry{densitecritique}{
  name={Densité critique},
  description={Valeur de densité correspondant au flux maximal, marquant la transition entre régime fluide et congestionné}
}

\newglossaryentry{calibration}{
  name={Calibration},
  description={Processus d'ajustement des paramètres d'un modèle pour reproduire au mieux les données observées}
}

\newglossaryentry{validation}{
  name={Validation},
  description={Processus d'évaluation de la qualité et de la pertinence d'un modèle à l'aide de données indépendantes de celles utilisées pour la calibration}
}

\newglossaryentry{rmse}{
  name={RMSE},
  description={Root Mean Square Error (Erreur Quadratique Moyenne), métrique utilisée pour évaluer la précision d'un modèle}
}

\newglossaryentry{geh}{
  name={GEH},
  description={Statistique de Geoffrey E. Havers couramment utilisée pour comparer les flux de trafic observés et modélisés}
}
 dans le préambule du document principal

\newglossaryentry{lwr}{
  name={Modèle LWR},
  description={Modèle macroscopique du trafic développé indépendamment par Lighthill et Whitham et par Richards. Il décrit l'évolution du trafic routier en utilisant une équation de conservation et une relation fondamentale entre densité, vitesse et flux}
}

\newglossaryentry{zemidjans}{
  name={Zémidjans},
  description={Moto-taxis au Bénin, constituant un mode de transport essentiel et une part importante du trafic routier urbain et péri-urbain}
}

\newglossaryentry{multiclasse}{
  name={Modèle multiclasses},
  description={Extension des modèles de trafic distinguant différentes catégories de véhicules, chacune avec ses propres caractéristiques (vitesse libre, taille, comportement)}
}

\newglossaryentry{gapfilling}{
  name={Gap-filling},
  description={Comportement caractéristique des motos consistant à occuper les espaces entre les véhicules plus grands, augmentant ainsi la densité effective du trafic}
}

\newglossaryentry{interweaving}{
  name={Interweaving},
  description={Comportement de circulation en zigzag des motos entre les files de véhicules, créant un flux transversal qui peut perturber l'écoulement des autres classes}
}

\newglossaryentry{coefralentissement}{
  name={Coefficient de ralentissement},
  description={Paramètre $\lambda_{\text{mat},i}$ qui module la vitesse maximale d'une classe de véhicule en fonction du type de revêtement routier}
}

\newglossaryentry{diagrammefondamental}{
  name={Diagramme fondamental},
  description={Représentation graphique de la relation entre la densité, la vitesse et le flux du trafic routier}
}

\newglossaryentry{ondechoc}{
  name={Onde de choc},
  description={Discontinuité se propageant dans le flux du trafic, généralement associée à la transition entre différents régimes de circulation}
}

\newglossaryentry{densitecritique}{
  name={Densité critique},
  description={Valeur de densité correspondant au flux maximal, marquant la transition entre régime fluide et congestionné}
}

\newglossaryentry{calibration}{
  name={Calibration},
  description={Processus d'ajustement des paramètres d'un modèle pour reproduire au mieux les données observées}
}

\newglossaryentry{validation}{
  name={Validation},
  description={Processus d'évaluation de la qualité et de la pertinence d'un modèle à l'aide de données indépendantes de celles utilisées pour la calibration}
}

\newglossaryentry{rmse}{
  name={RMSE},
  description={Root Mean Square Error (Erreur Quadratique Moyenne), métrique utilisée pour évaluer la précision d'un modèle}
}

\newglossaryentry{geh}{
  name={GEH},
  description={Statistique de Geoffrey E. Havers couramment utilisée pour comparer les flux de trafic observés et modélisés}
}
 dans le préambule du document principal

\newglossaryentry{lwr}{
  name={Modèle LWR},
  description={Modèle macroscopique du trafic développé indépendamment par Lighthill et Whitham et par Richards. Il décrit l'évolution du trafic routier en utilisant une équation de conservation et une relation fondamentale entre densité, vitesse et flux}
}

\newglossaryentry{zemidjans}{
  name={Zémidjans},
  description={Moto-taxis au Bénin, constituant un mode de transport essentiel et une part importante du trafic routier urbain et péri-urbain}
}

\newglossaryentry{multiclasse}{
  name={Modèle multiclasses},
  description={Extension des modèles de trafic distinguant différentes catégories de véhicules, chacune avec ses propres caractéristiques (vitesse libre, taille, comportement)}
}

\newglossaryentry{gapfilling}{
  name={Gap-filling},
  description={Comportement caractéristique des motos consistant à occuper les espaces entre les véhicules plus grands, augmentant ainsi la densité effective du trafic}
}

\newglossaryentry{interweaving}{
  name={Interweaving},
  description={Comportement de circulation en zigzag des motos entre les files de véhicules, créant un flux transversal qui peut perturber l'écoulement des autres classes}
}

\newglossaryentry{coefralentissement}{
  name={Coefficient de ralentissement},
  description={Paramètre $\lambda_{\text{mat},i}$ qui module la vitesse maximale d'une classe de véhicule en fonction du type de revêtement routier}
}

\newglossaryentry{diagrammefondamental}{
  name={Diagramme fondamental},
  description={Représentation graphique de la relation entre la densité, la vitesse et le flux du trafic routier}
}

\newglossaryentry{ondechoc}{
  name={Onde de choc},
  description={Discontinuité se propageant dans le flux du trafic, généralement associée à la transition entre différents régimes de circulation}
}

\newglossaryentry{densitecritique}{
  name={Densité critique},
  description={Valeur de densité correspondant au flux maximal, marquant la transition entre régime fluide et congestionné}
}

\newglossaryentry{calibration}{
  name={Calibration},
  description={Processus d'ajustement des paramètres d'un modèle pour reproduire au mieux les données observées}
}

\newglossaryentry{validation}{
  name={Validation},
  description={Processus d'évaluation de la qualité et de la pertinence d'un modèle à l'aide de données indépendantes de celles utilisées pour la calibration}
}

\newglossaryentry{rmse}{
  name={RMSE},
  description={Root Mean Square Error (Erreur Quadratique Moyenne), métrique utilisée pour évaluer la précision d'un modèle}
}

\newglossaryentry{geh}{
  name={GEH},
  description={Statistique de Geoffrey E. Havers couramment utilisée pour comparer les flux de trafic observés et modélisés}
}
