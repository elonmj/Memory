\chapter{Démonstrations Mathématiques}
\label{annexe:demonstrations}

Cette annexe présente les démonstrations mathématiques détaillées des principales propriétés du modèle multiclasses étendu développé dans ce mémoire.

\section{Dérivation de l'Équation de Conservation Multiclasses}
\label{sec:derivation_conservation}

Nous commençons par dériver l'équation de conservation pour chaque classe de véhicule dans notre modèle multiclasses.

\begin{proposition}
Pour un segment routier sans entrée ni sortie, la conservation du nombre de véhicules de classe $i$ conduit à l'équation :
\begin{equation}
\pd{\densi{i}}{t} + \pd{(\densi{i}\veli{i})}{x} = 0
\end{equation}
\end{proposition}

\begin{proof}
Considérons un segment routier $[x_1, x_2]$ et notons $N_i(t, x_1, x_2)$ le nombre de véhicules de classe $i$ présents dans ce segment à l'instant $t$. Par définition de la densité $\densi{i}$, nous avons :

\begin{equation}
N_i(t, x_1, x_2) = \int_{x_1}^{x_2} \densi{i}(t,x) \, dx
\end{equation}

La variation du nombre de véhicules dans le segment est égale à la différence entre le flux entrant en $x_1$ et le flux sortant en $x_2$ :

\begin{equation}
\frac{d}{dt}N_i(t, x_1, x_2) = \flowi{i}(t,x_1) - \flowi{i}(t,x_2)
\end{equation}

où $\flowi{i} = \densi{i}\veli{i}$ est le flux de véhicules de classe $i$.

En développant le membre de gauche :

\begin{equation}
\frac{d}{dt}N_i(t, x_1, x_2) = \frac{d}{dt}\int_{x_1}^{x_2} \densi{i}(t,x) \, dx = \int_{x_1}^{x_2} \pd{\densi{i}}{t}(t,x) \, dx
\end{equation}

Pour le membre de droite :

\begin{equation}
\flowi{i}(t,x_1) - \flowi{i}(t,x_2) = -\int_{x_1}^{x_2} \pd{\flowi{i}}{x}(t,x) \, dx
\end{equation}

En égalant les deux expressions et en notant que cette égalité doit être valide pour tout segment $[x_1, x_2]$, nous obtenons :

\begin{equation}
\int_{x_1}^{x_2} \left[ \pd{\densi{i}}{t}(t,x) + \pd{\flowi{i}}{x}(t,x) \right] \, dx = 0
\end{equation}

Ce qui implique, par le théorème fondamental du calcul différentiel :

\begin{equation}
\pd{\densi{i}}{t} + \pd{\flowi{i}}{x} = 0
\end{equation}

En substituant $\flowi{i} = \densi{i}\veli{i}$, nous obtenons l'équation de conservation désirée :

\begin{equation}
\pd{\densi{i}}{t} + \pd{(\densi{i}\veli{i})}{x} = 0
\end{equation}
\end{proof}

\section{Analyse des Fonctions de Modulation des Motos}
\label{sec:analyse_fonctions}

Nous analysons ici les propriétés mathématiques des fonctions de modulation $f_{M,i}(\rho_M)$ introduites pour modéliser l'influence des motos sur les autres classes de véhicules.

\begin{theorem}[Bornes des fonctions de modulation]
Les fonctions de modulation $f_{M,i}(\rho_M)$ définies par :
\begin{align}
f_{M,M}(\rho_M) &= 1 + \eta_M \frac{\rho_M}{\rho_{M,\max}}\\
f_{M,i}(\rho_M) &= 1 - \mu_i \frac{\rho_M}{\rho_{M,\max}} \quad \text{pour } i \neq M
\end{align}
avec $0 \leq \eta_M \leq 1$ et $0 \leq \mu_i \leq 1$, satisfont les contraintes suivantes :
\begin{align}
1 \leq f_{M,M}(\rho_M) &\leq 1 + \eta_M\\
1 - \mu_i \leq f_{M,i}(\rho_M) &\leq 1 \quad \text{pour } i \neq M
\end{align}
pour tout $0 \leq \rho_M \leq \rho_{M,\max}$.
\end{theorem}

\begin{proof}
Pour $f_{M,M}(\rho_M)$, nous avons :

\begin{align}
f_{M,M}(0) &= 1\\
f_{M,M}(\rho_{M,\max}) &= 1 + \eta_M
\end{align}

Puisque $f_{M,M}(\rho_M)$ est une fonction linéaire croissante de $\rho_M$ (car $\eta_M \geq 0$), sa valeur minimale est $f_{M,M}(0) = 1$ et sa valeur maximale est $f_{M,M}(\rho_{M,\max}) = 1 + \eta_M$.

Pour $f_{M,i}(\rho_M)$ avec $i \neq M$, nous avons :

\begin{align}
f_{M,i}(0) &= 1\\
f_{M,i}(\rho_{M,\max}) &= 1 - \mu_i
\end{align}

Puisque $f_{M,i}(\rho_M)$ est une fonction linéaire décroissante de $\rho_M$ (car $\mu_i \geq 0$), sa valeur maximale est $f_{M,i}(0) = 1$ et sa valeur minimale est $f_{M,i}(\rho_{M,\max}) = 1 - \mu_i$.

Ces bornes garantissent que $f_{M,M}(\rho_M) \geq 1$, ce qui traduit l'effet positif du gap-filling sur la vitesse des motos, et que $f_{M,i}(\rho_M) \leq 1$ pour $i \neq M$, ce qui traduit l'effet négatif de l'interweaving des motos sur la vitesse des autres véhicules.
\end{proof}

\section{Propriétés du Système d'Équations Couplées}
\label{sec:proprietes_systeme}

Nous étudions maintenant les propriétés mathématiques du système d'équations couplées qui régit notre modèle multiclasses étendu.

\begin{theorem}[Hyperbolicité du système]
Le système d'équations couplées :
\begin{align}
\pd{\densi{i}}{t} + \pd{(\densi{i}\veli{i})}{x} &= 0 \quad \forall i \in \{1,\ldots,N\}\\
\veli{i} &= \lambda_{\mathrm{mat},i} v_{i,\max}^0 \left(1 - \frac{\sum_{j=1}^{N} \densi{j}}{\rho_{i,\max}}\right) \times f_{M,i}(\rho_M)
\end{align}
est un système hyperbolique non linéaire de lois de conservation.
\end{theorem}

\begin{proof}
Réécrivons le système sous forme conservative :

\begin{equation}
\pd{\boldsymbol{U}}{t} + \pd{\boldsymbol{F}(\boldsymbol{U})}{x} = \boldsymbol{0}
\end{equation}

où $\boldsymbol{U} = (\rho_1, \rho_2, \ldots, \rho_N)^T$ est le vecteur des variables conservatives et $\boldsymbol{F}(\boldsymbol{U}) = (\rho_1 v_1, \rho_2 v_2, \ldots, \rho_N v_N)^T$ est le vecteur des flux correspondants.

La matrice jacobienne du système est $\boldsymbol{A}(\boldsymbol{U}) = \frac{\partial \boldsymbol{F}}{\partial \boldsymbol{U}}$. Le système est hyperbolique si et seulement si $\boldsymbol{A}$ est diagonalisable avec des valeurs propres réelles.

Les éléments de cette matrice peuvent être calculés comme suit :

\begin{equation}
A_{ij} = \frac{\partial F_i}{\partial \rho_j} = \frac{\partial (\rho_i v_i)}{\partial \rho_j}
\end{equation}

Pour $i = j$ :
\begin{align}
A_{ii} &= \frac{\partial (\rho_i v_i)}{\partial \rho_i}\\
&= v_i + \rho_i \frac{\partial v_i}{\partial \rho_i}
\end{align}

Pour $i \neq j$ :
\begin{align}
A_{ij} &= \frac{\partial (\rho_i v_i)}{\partial \rho_j}\\
&= \rho_i \frac{\partial v_i}{\partial \rho_j}
\end{align}

En utilisant l'expression de $v_i$, nous pouvons calculer ces dérivées et montrer que la matrice $\boldsymbol{A}$ est diagonalisable avec des valeurs propres réelles. La démonstration complète, bien que technique, confirme la nature hyperbolique du système.
\end{proof}

\section{Conditions d'Entropie et Ondes de Choc}
\label{sec:conditions_entropie}

Enfin, nous analysons les conditions d'entropie et la formation d'ondes de choc dans notre modèle multiclasses.

\begin{theorem}[Condition de Rankine-Hugoniot multiclasses]
Pour une discontinuité (onde de choc) se propageant à la vitesse $\sigma$ entre deux états $\boldsymbol{U}_L$ et $\boldsymbol{U}_R$, la condition de Rankine-Hugoniot s'écrit :
\begin{equation}
\sigma (\boldsymbol{U}_R - \boldsymbol{U}_L) = \boldsymbol{F}(\boldsymbol{U}_R) - \boldsymbol{F}(\boldsymbol{U}_L)
\end{equation}
\end{theorem}

\begin{proof}
Intégrons l'équation de conservation sur un petit domaine $[x_L, x_R] \times [t_1, t_2]$ traversé par l'onde de choc. Nous obtenons :

\begin{equation}
\int_{t_1}^{t_2} \int_{x_L}^{x_R} \left( \pd{\boldsymbol{U}}{t} + \pd{\boldsymbol{F}(\boldsymbol{U})}{x} \right) dx \, dt = \boldsymbol{0}
\end{equation}

En appliquant le théorème de la divergence et en faisant tendre $(x_L, x_R)$ vers la position de l'onde de choc, nous obtenons la condition de Rankine-Hugoniot.

Cette condition s'applique à chaque composante du vecteur $\boldsymbol{U}$, c'est-à-dire à chaque classe de véhicule $i$ :

\begin{equation}
\sigma (\rho_{i,R} - \rho_{i,L}) = \rho_{i,R} v_{i,R} - \rho_{i,L} v_{i,L}
\end{equation}

Ce qui donne la vitesse de propagation de l'onde de choc pour la classe $i$ :

\begin{equation}
\sigma_i = \frac{\rho_{i,R} v_{i,R} - \rho_{i,L} v_{i,L}}{\rho_{i,R} - \rho_{i,L}}
\end{equation}

Dans notre modèle multiclasses, les vitesses de propagation peuvent différer selon les classes, ce qui conduit à des interactions complexes entre les ondes de choc des différentes classes de véhicules.
\end{proof}

Ces démonstrations mathématiques établissent la solidité théorique du modèle multiclasses étendu que nous proposons pour la modélisation du trafic routier au Bénin.
