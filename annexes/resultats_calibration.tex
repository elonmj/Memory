\chapter{Résultats Détaillés de Calibration}
\label{annexe:resultats_calibration}

Cette annexe présente les résultats détaillés du processus de calibration du modèle multiclasses étendu pour le trafic béninois, en complément des résultats synthétiques présentés dans le chapitre \ref{chap:calibrage_validation}.

\section{Paramètres Calibrés pour Chaque Type de Route}
\label{sec:parametres_routes}

\subsection{Routes Bitumées en Bon État}
\label{subsec:bitumees_bon}

\begin{table}[htbp]
\centering
\caption{Paramètres calibrés pour les routes bitumées en bon état}
\label{tab:calibration_bitumees_bon}
\begin{tabular}{lccccc}
\toprule
\textbf{Classe de véhicule} & $v_{i,\max}^0$ (km/h) & $\rho_{i,\max}$ (véh/km) & $\lambda_{\text{mat},i}$ & $\eta_M$ & $\mu_i$ \\
\midrule
Motos & $60.2 \pm 3.4$ & $242.6 \pm 12.8$ & $1.00 \pm 0.03$ & $0.42 \pm 0.07$ & --- \\
Voitures particulières & $71.5 \pm 4.2$ & $180.3 \pm 9.6$ & $0.95 \pm 0.04$ & --- & $0.31 \pm 0.05$ \\
Taxis & $65.7 \pm 3.8$ & $179.8 \pm 8.9$ & $0.92 \pm 0.03$ & --- & $0.38 \pm 0.06$ \\
Bus & $54.9 \pm 3.1$ & $140.2 \pm 7.4$ & $0.90 \pm 0.04$ & --- & $0.51 \pm 0.08$ \\
Camions & $49.6 \pm 2.9$ & $120.5 \pm 6.5$ & $0.88 \pm 0.05$ & --- & $0.62 \pm 0.09$ \\
\bottomrule
\end{tabular}
\end{table}

\subsection{Routes Bitumées Dégradées}
\label{subsec:bitumees_degradees}

\begin{table}[htbp]
\centering
\caption{Paramètres calibrés pour les routes bitumées dégradées}
\label{tab:calibration_bitumees_degradees}
\begin{tabular}{lccccc}
\toprule
\textbf{Classe de véhicule} & $v_{i,\max}^0$ (km/h) & $\rho_{i,\max}$ (véh/km) & $\lambda_{\text{mat},i}$ & $\eta_M$ & $\mu_i$ \\
\midrule
Motos & $60.2 \pm 3.4$ & $238.4 \pm 13.2$ & $0.82 \pm 0.06$ & $0.45 \pm 0.08$ & --- \\
Voitures particulières & $71.5 \pm 4.2$ & $177.6 \pm 10.1$ & $0.68 \pm 0.07$ & --- & $0.33 \pm 0.06$ \\
Taxis & $65.7 \pm 3.8$ & $176.5 \pm 9.4$ & $0.70 \pm 0.06$ & --- & $0.40 \pm 0.07$ \\
Bus & $54.9 \pm 3.1$ & $138.7 \pm 8.2$ & $0.65 \pm 0.08$ & --- & $0.54 \pm 0.09$ \\
Camions & $49.6 \pm 2.9$ & $119.3 \pm 7.8$ & $0.60 \pm 0.09$ & --- & $0.65 \pm 0.10$ \\
\bottomrule
\end{tabular}
\end{table}

\subsection{Routes en Terre Compactée}
\label{subsec:terre_compactee}

\begin{table}[htbp]
\centering
\caption{Paramètres calibrés pour les routes en terre compactée}
\label{tab:calibration_terre}
\begin{tabular}{lccccc}
\toprule
\textbf{Classe de véhicule} & $v_{i,\max}^0$ (km/h) & $\rho_{i,\max}$ (véh/km) & $\lambda_{\text{mat},i}$ & $\eta_M$ & $\mu_i$ \\
\midrule
Motos & $60.2 \pm 3.4$ & $231.5 \pm 14.6$ & $0.76 \pm 0.08$ & $0.48 \pm 0.09$ & --- \\
Voitures particulières & $71.5 \pm 4.2$ & $172.8 \pm 11.3$ & $0.51 \pm 0.09$ & --- & $0.37 \pm 0.08$ \\
Taxis & $65.7 \pm 3.8$ & $171.2 \pm 10.8$ & $0.53 \pm 0.08$ & --- & $0.43 \pm 0.09$ \\
Bus & $54.9 \pm 3.1$ & $135.4 \pm 9.5$ & $0.48 \pm 0.10$ & --- & $0.58 \pm 0.11$ \\
Camions & $49.6 \pm 2.9$ & $116.7 \pm 8.9$ & $0.45 \pm 0.11$ & --- & $0.68 \pm 0.12$ \\
\bottomrule
\end{tabular}
\end{table}

\subsection{Routes Pavées}
\label{subsec:pavees}

\begin{table}[htbp]
\centering
\caption{Paramètres calibrés pour les routes pavées}
\label{tab:calibration_pavees}
\begin{tabular}{lccccc}
\toprule
\textbf{Classe de véhicule} & $v_{i,\max}^0$ (km/h) & $\rho_{i,\max}$ (véh/km) & $\lambda_{\text{mat},i}$ & $\eta_M$ & $\mu_i$ \\
\midrule
Motos & $60.2 \pm 3.4$ & $235.7 \pm 13.9$ & $0.80 \pm 0.07$ & $0.46 \pm 0.08$ & --- \\
Voitures particulières & $71.5 \pm 4.2$ & $175.2 \pm 10.7$ & $0.62 \pm 0.08$ & --- & $0.35 \pm 0.07$ \\
Taxis & $65.7 \pm 3.8$ & $174.3 \pm 10.2$ & $0.64 \pm 0.07$ & --- & $0.42 \pm 0.08$ \\
Bus & $54.9 \pm 3.1$ & $137.1 \pm 8.8$ & $0.58 \pm 0.09$ & --- & $0.56 \pm 0.10$ \\
Camions & $49.6 \pm 2.9$ & $117.9 \pm 8.3$ & $0.55 \pm 0.10$ & --- & $0.66 \pm 0.11$ \\
\bottomrule
\end{tabular}
\end{table}

\section{Courbes de Convergence de la Calibration}
\label{sec:convergence_calibration}

\subsection{Évolution de la Fonction Objectif}
\label{subsec:fonction_objectif}

La figure \ref{fig:convergence_objectif} montre l'évolution de la fonction objectif (RMSE) au cours du processus d'optimisation pour différents types de routes.

\begin{figure}[htbp]
\centering
\includegraphics[width=0.9\textwidth]{images/appendices/convergence_objectif}
\caption{Évolution de la fonction objectif (RMSE) en fonction des itérations pour différents types de routes.}
\label{fig:convergence_objectif}
\end{figure}

On observe une convergence rapide dans les premières itérations, suivie d'une stabilisation progressive. La convergence est plus lente pour les routes en terre, reflétant la plus grande variabilité des données pour ce type de route.

\subsection{Convergence des Paramètres Clés}
\label{subsec:convergence_parametres}

Les figures \ref{fig:convergence_lambda} et \ref{fig:convergence_eta_mu} montrent l'évolution des paramètres $\lambda_{\text{mat},i}$ et $\eta_M, \mu_i$ au cours du processus de calibration.

\begin{figure}[htbp]
\centering
\includegraphics[width=0.9\textwidth]{images/appendices/convergence_lambda}
\caption{Évolution des coefficients de ralentissement $\lambda_{\text{mat},i}$ au cours de la calibration.}
\label{fig:convergence_lambda}
\end{figure}

\begin{figure}[htbp]
\centering
\includegraphics[width=0.9\textwidth]{images/appendices/convergence_eta_mu}
\caption{Évolution des paramètres de modulation $\eta_M$ et $\mu_i$ au cours de la calibration.}
\label{fig:convergence_eta_mu}
\end{figure}

\section{Analyse de Corrélation entre Paramètres}
\label{sec:correlation_parametres}

L'analyse des corrélations entre les différents paramètres calibrés permet d'identifier les dépendances potentielles et d'évaluer la robustesse du processus de calibration.

\begin{figure}[htbp]
\centering
\includegraphics[width=0.9\textwidth]{images/appendices/matrice_correlation}
\caption{Matrice de corrélation entre les principaux paramètres calibrés.}
\label{fig:matrice_correlation}
\end{figure}

Les observations principales sont :
\begin{itemize}
\item Forte corrélation négative ($r = -0.78$) entre $\lambda_{\text{mat},i}$ et $\mu_i$ pour les voitures, suggérant que ces deux effets se compensent partiellement.
\item Corrélation modérée ($r = 0.51$) entre $\eta_M$ et $\rho_{M,\max}$, indiquant une interaction entre la densité maximale des motos et leur capacité de gap-filling.
\item Faible corrélation entre les paramètres de différentes classes de véhicules, confirmant la pertinence de l'approche multiclasses.
\end{itemize}

\section{Validation Croisée}
\label{sec:validation_croisee}

Pour évaluer la robustesse du modèle calibré, une validation croisée à $k=5$ plis a été réalisée.

\begin{table}[htbp]
\centering
\caption{Résultats de la validation croisée à 5 plis}
\label{tab:validation_croisee}
\begin{tabular}{lcccc}
\toprule
\textbf{Type de route} & \textbf{RMSE moyen} & \textbf{MAE moyen} & \textbf{$R^2$ moyen} & \textbf{GEH moyen} \\
\midrule
Bitumée (bon état) & 4.2 \% & 3.8 \% & 0.92 & 2.7 \\
Bitumée (dégradée) & 5.7 \% & 5.1 \% & 0.88 & 3.5 \\
Terre compactée & 8.3 \% & 7.4 \% & 0.81 & 4.8 \\
Pavée & 6.2 \% & 5.5 \% & 0.85 & 3.9 \\
\bottomrule
\end{tabular}
\end{table}

La faible variabilité des métriques d'erreur entre les différents plis (écart-type relatif < 10\%) indique une bonne généralisation du modèle.

\section{Comparaison avec d'Autres Modèles}
\label{sec:comparaison_modeles}

Cette section compare les performances du modèle étendu proposé avec celles d'autres approches de modélisation du trafic.

\begin{table}[htbp]
\centering
\caption{Comparaison des performances avec d'autres modèles}
\label{tab:comparaison_modeles}
\begin{tabular}{lccc}
\toprule
\textbf{Modèle} & \textbf{RMSE moyen} & \textbf{$R^2$ moyen} & \textbf{GEH moyen} \\
\midrule
LWR standard & 12.7 \% & 0.68 & 7.9 \\
Multiclasses sans $\lambda_{\text{mat},i}$ & 8.5 \% & 0.79 & 5.3 \\
Multiclasses sans $f_{M,i}(\rho_M)$ & 7.2 \% & 0.83 & 4.6 \\
\textbf{Notre modèle étendu complet} & \textbf{5.3 \%} & \textbf{0.90} & \textbf{3.2} \\
\bottomrule
\end{tabular}
\end{table}

Ces résultats confirment l'amélioration significative apportée par chaque composante de notre extension du modèle LWR, avec une réduction de plus de 58\% de l'erreur RMSE par rapport au modèle standard.