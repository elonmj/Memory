\chapter*{Liste des Abréviations}
\addcontentsline{toc}{chapter}{Liste des Abréviations}
\thispagestyle{fancy}

\begin{tabular}{ll}
\toprule
\textbf{Abréviation} & \textbf{Signification} \\
\midrule
EDP & Équation aux Dérivées Partielles \\
GEH & Geoffrey E. Havers (statistique de validation du trafic) \\
IMSP & Institut de Mathématiques et de Sciences Physiques \\
LWR & Lighthill-Whitham-Richards (modèle macroscopique de trafic) \\
MAE & Mean Absolute Error (Erreur Absolue Moyenne) \\
MC & Monte Carlo (méthode de simulation) \\
RMSE & Root Mean Square Error (Erreur Quadratique Moyenne) \\
SIG & Système d'Information Géographique \\
UAC & Université d'Abomey-Calavi \\
véh/km & Véhicules par kilomètre (unité de densité de trafic) \\
véh/h & Véhicules par heure (unité de flux de trafic) \\
CFL & Courant-Friedrichs-Lewy \\
CTM & Cell Transmission Model (Modèle de Transmission Cellulaire) \\
PDE & Partial Differential Equation \\
HOV & High Occupancy Vehicle (Véhicule à Occupation Élevée) \\
HLL & Harten-Lax-van Leer (solveur numérique) \\
WENO & Weighted Essentially Non-Oscillatory (schéma numérique) \\
TVD & Total Variation Diminishing (schéma à variation totale décroissante) \\
INSAE & Institut National de la Statistique et de l'Analyse Économique (Bénin) \\
AERC & African Economic Research Consortium \\
\bottomrule
\end{tabular}

\vspace{1cm}

\begin{tabular}{ll}
\toprule
\textbf{Symbole} & \textbf{Description} \\
\midrule
$\rho, \dens$ & Densité du trafic \\
$v, \vel$ & Vitesse du trafic \\
$q, \flow$ & Flux du trafic \\
$\rho_i, \densi{i}$ & Densité de la classe $i$ de véhicules \\
$\rho_M, \densM$ & Densité des motos \\
$\lambda_{\text{mat},i}$ & Coefficient de ralentissement lié au revêtement pour la classe $i$ \\
$f_{M,i}$ & Fonction de modulation de l'effet des motos sur la classe $i$ \\
$\eta_M$ & Coefficient de gap-filling pour les motos \\
$\mu_i$ & Coefficient d'interweaving pour la classe $i$ \\
$S_i(x,t)$ & Terme source/puits pour la classe $i$ à la position $x$ et au temps $t$ \\
$\Delta q$ & Variation du flux à une intersection \\
\bottomrule
\end{tabular}
